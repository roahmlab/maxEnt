\documentclass{article}
\usepackage{relsize}
\usepackage{verbatim}
\newcommand{\spotless}{{SPOT{\relsize{-2}LESS}}}
\newcommand{\matlab}{MATLAB}
\title{\spotless{}}
\date{April 22nd, 2012}
\author{Mark M. Tobenkin, Alexandre Megretski}
\begin{document}
\maketitle
\begin{abstract}
  \spotless{} is a tool for authors of \matlab{} toolboxes which rely on conic optimization.  It provides tools for posing problems to modern interior point SDP solvers such as SeDuMi \cite{SeDuMi} and SDPT3 \cite{SDPT3}.  The design of the toolbox is focused on providing ease for manipulating the objects involved in a problem definition programmatically, and efficiency in creating many related conic constraints simultaneously.  \spotless{} consists of a reduced symbolic polynomial library which uses floating point representation of coefficients and an engine for constructing conic programs represented by these expressions.
\end{abstract}
\section{Manipulating Polynomials: {\tt msspoly}}
The class {\tt msspoly} stores and manipulates multivartie matrix polynomials.  {\tt msspoly} are constructed in one of three ways.

\begin{verbatim}
x = msspoly('x');
y = msspoly('y',3);
z = msspoly('z',[3 4]);
\end{verbatim}

Here {\tt x} will be an {\tt 1-by-1} matrix with a single scalar element whose name is {\tt 'x'}.


The following functions behave much as would be expected in any symbolic library:
\begin{enumerate}
\item Arithmetic: {\tt diag, imag, minus, mpower, mtimes, plus, power, real, subs, sum, times, trace, uminus, uplus}.
\item Array Manipulation: {\tt ctranspose, horzcat, isempty, length, repmat, reshape, size, subsasgn, subsref, transpose, vertcat}.
\end{enumerate}
A caveat is that {\tt subsasgn} will not expand the size of a 
\section{Constructing Conic Programs}

\end{document}